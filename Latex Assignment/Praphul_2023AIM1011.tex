\documentclass{article}
\usepackage{graphicx} % Required for inserting images
\usepackage[margin=1in]{geometry}
\usepackage{multicol}
\usepackage{amsmath}
\usepackage{algorithm}
\usepackage{algpseudocode}
\usepackage{subcaption}






\title{Joint Optimization of Trajectory Planning and Task Scheduling in
Heterogeneous Multi-UAV System}
\author{Praphul P Thekkila}
\date{August 2023}

\begin{document}


\maketitle
\begin{multicols}{2}
\section*{Abstract}
he use of unmanned aerial vehicles (UAV) as a new sensing paradigm is emerging for surveillance and tracking
applications, especially in the infrastructure-less environment. One such application of UAVs is in the construction industry where currently prevalent manual progress
tracking results in schedule delays and cost overruns. In
this paper, we develop a heterogeneous multi-UAV framework for progress tracking of large construction sites. The
proposed framework consists of Edge UAV which coordinates the data relay of the visual sensor-equipped Inspection UAVs (I UAV s) to the cloud. Our framework jointly
takes into consideration the trajectory optimization of the
Edge UAV and the stability of system queues. In particular, we develop a Distance and Access Latency Aware
Trajectory (DLAT) optimization that generates a fair access schedule for I UAV s. In addition, a Lyapunov-based
online optimization ensures the system stability of the average queue backlogs for data offloading tasks. Through a
message based mechanism, the coordination between the
set of I UAV s and Edge UAV is ensured without any
dependence on any central entity or message broadcasts.
The performance of our proposed framework with joint
optimization algorithm is validated by extensive simulation results in different parameter settings

\subsection*{Keyword:} Path Planning, Task Scheduling, Data Off loading, Construction Site Monitoring, Unmanned Aerial
Vehicles (UAVs), Lyapunov Optimization


\section{Introduction}
The unmanned aerial vehicle (UAV) based solutions are
emerging in various domains such as wireless sensing [1],
payload delivery \cite{ruggiero2018aerial}, precision agriculture \cite{boursianis2020internet},, help and
rescue operations  \cite{waharte2010supporting},etc.    Moreover, with the current
trend of automation, sensing and information exchange in Industry 4.0, UAV based applications are also finding
their place in the construction industry especially for resource tracking and progress monitoring using aerial imagery. Such solutions are helpful in infrastructure-less
large construction sites as they provide ease of deployment, quick access to the ground-truth data and higher
reachability and coverage \cite{hamledari2018uav}.  Further, the autonomous
or semi-autonomous UAV based solutions could facilitate
progress monitoring, building inspections (for cracks or
other defects), safety inspections (to find any environmental hazards) and many more construction-specific audits automatically. The UAV based visual monitoring of
under-construction projects also allows simultaneous observability of ground-truth data by different collaborating
entities. Availability of such data and information helps
in timely assessments that could reduce schedule delays,
cost overruns, resource wastage and financial losses which
are not uncommon in construction projects.



A plausible solution to address the aforementioned
challenges could be a Mobile Edge Computing (MEC) \cite{mao2017survey} based heterogeneous multi-UAV framework. Such a
framework along with the prior geometric knowledge available about the construction site as gathered from a Building Information Model (BIM) \cite{golparvar2011integrated} could help create an effective multi-UAV based visual monitoring system for construction sites. As for any constrained environment, the
optimization of computational resources is central to develop a solution. The integration of UAVs and MEC into
a single framework could facilitate that with efficient data
collection/processing from the UAV based dynamic sensors in infrastructure-less environments \cite{nguyen2020towards} In addition,
an MEC based framework can help to perform partial
computation offloading wherein a part of data is processed
by the UAVs while the rest gets offloaded to the cloud.



An MEC based UAV framework is not new and the deployment of the UAVs as base stations or edge servers
is widely studied [9, 10]. These studies reflect on the flexibility in deployment of UAV based edge computing
components. However, there is a problem of buffer overflow of UAVs due to the limited on-board processing and
the shared bandwidth to transfer data to the cloud which
leads to instability in the system. In addition, the dynamic nature of such systems with varying data traffic
and continuous movement of UAVs makes it difficult to
stabilize or control the system in a deterministic manner. Researchers have used online Lyapunov optimization \cite{neely2010stochastic}to address such system instabilities. Lyapunov optimization considers the stability of the system with time
varying data and optimizes time averages of system utility
and queue backlogs. to address such system instabilities.
and queue backlogs.


In this paper, we address the challenges of deploying
a heterogeneous multi-UAV system for construction site
monitoring by the joint optimization of UAV trajectory
planning and data offloading task scheduling. The proposed framework employs two types of UAVs viz. Inspection UAVs I UAV s and Mobile Edge UAVs (Edge UAV ).
While the former is deployed as visual sensors to collect
visual data from different locations of the site, the latter
interacts and collects data from I UAV s, and offloads the
same to the cloud. The core objective of the framework
is to minimize the total energy consumption of the system while considering the data queue backlogs of I UAV s
and Edge UAV and also jointly optimizing the trajectory
of the Edge UAV in accordance with the trajectories of
I UAV s having minimum access latency and travel distance. The online resource management such as transmission power and processor frequency of the Edge UAV is
evolved using Lyapunov optimization (as in [12]).
The rest of the paper is organised as follows: Section
2 presents the proposed heterogeneous multi-UAV framework for construction site monitoring. The overall system
objective is discussed in Sections 3. Sections 4 and 5 discuss the trajectory optimization and Lyapunov based system stability, respectively. The simulation setup has been presented in Section 6. Section 7 discusses the results
gathered from the experiments while Section 8 concludes
the paper.

\section{Heterogeneous Multi-UAV Frame-work}
Figure ?? depicts the overall multi-UAV framework with
all its components. The system consists of two hetero-geneous UAVs i.e. a set of Inspection UAVs I UAV =
{I UAV1, I UAV2, I UAV3, ....., I UAVN } and a Mobile
Edge UAV (Edge UAV ). I UAV s are smaller in size and
are more agile. They collect visual data from a set of
Point of Interests (PoIs) denoted as L = {l1, l2, l3....lk}
across the construction site. As the construction sites are
infrastructure-less environments, there are limited Access
Points (AP) available for connectivity to the cloud. Further, the I UAV s possess limited connectivity range that
makes it difficult for them to transfer data to cloud directly. In addition, the I UAV s move in the 3D Cartesian
coordinate system. The Edge UAV , which is larger in size
and possesses higher computational capabilities, coordinates with the I UAV s to relay the data (after partially
processing the same) to the cloud. Edge UAV always
maintains a constant height and thus its trajectory lies in
an horizontal plane.
The communication between I UAV and Edge UAV
(A2A channel) has limited range and bandwidth. We
have assumed the achievable data transmission rate of
the I UAVi
in a given time slot as d
of f
i
(t). Further, The
height of the Edge UAV is h which is dependent on coverage range r and line of sight (LoS) loss caused due to
environmental effects [13].
The A2A channel power gain (\zeta) from I_UAV to Edge UAV can be given as:

\[\zeta = g_o* (dis_0/dis_t)^\theta\]

where $g_0$ is the path loss constant,$d_0$ is the reference
distance,$dis_t$ distance between the UAVs and \theta is the path loss exponent.

\subsection{ Data collection and offloading}

Each PoI (li) is a tuple (< di
, ψi >) where di specifies the
amount of data (images) to be collected and ψi denotes the
coordinates of the site locations in 3D space. The sequence
of PoIs to be visited is provided to I UAV s and same is
also shared with the Edge UAV . During the traversal
along the sequence of PoIs, the limited buffer may make
the I UAV wait at some PoIs along the trajectory until
it offloads the data to the Edge UAV .
The Edge UAV can communicate with one of the
I UAVi
in a time slot. The data gathered by each of
the I UAVi
in a time slot t is denoted by Ai(t). Qi(t)
represents the queue of the I UAVi and $d_i^off$ denotes the the amount of data offloaded to the Edge UAV by the
I UAVi
in time-slot t. The recursive equation to update
the Qi(t) is as follows:.

\[Q_i(t+1) =max{}Q_i(t) - d_i^off (t),0}  + A_i(t)\]

The Edge UAV accepts data from the selected I UAVi
in the time-slot t in its queue L(t). The following equation
updates L(t) recursively:

\[L(t+1) = max{L(t) - c(t) - d_edge^off (t),0 +A_edge (t)\] where $A_edge$  is the data arrived from the selected
I UAVi
in time-slot t, c(t) is the data processed by the
Edge UAV in time-slot t, and $d_off ^edge$ is is the number of
bits offloaded to the cloud in time-slot t.

\section{System Objective}

In the proposed framework, the offloading of data happens
at two stages - 1) from I UAVi to Edge UAV and 2) from
Edge UAV to the cloud. Our main focus is to achieve the
end-to-end data offloading to the cloud by minimizing the
total energy consumption of the whole system (Esystem)
which is defined as:
\begin{equation}
    E_{system} (t) = E_{edge}^{transition} (t) +
    
     E_{edge}^{Comm} (t) + (\sum_{i=1}^{n} (E_i^{Comm} (t)))
\end{equation}

\subsubsection{Transition Energy of Edge-UAV}

The transition energy of Edge UAV refers to the energy
consumed in moving from one location to another. The
transition energy of the Edge UAV is given as:
\begin{equation}
    E_{edge}^{transition} = K||vel(t)||^2
\end{equation}
where K is a constant that depends on the total mass of
the Edge UAV and vel(t) is the velocity of I UAV

\subsubsection{Communication Energy of Edge UAV}
Edge UAV offloads the data to the cloud through a wireless channel [14]. The communication energy consumed
to transmit the data to the cloud is given as:

\begin{equation}
    E_{edge}^{comm}(t) = (2^{d^{off}_edge (t)/W*\tau} - 1)* N_0W/\zeta * \tau
\end{equation}

\subsubsection{Communication Energy of I UAV}
The energy consumed for offloading the d
of f
i
(t) data bits
at time slot t from the selected I UAVi to the Edge UAV
using the A2A channel of bandwidth W Hz is given similarly to Equation 6 as:

\begin{equation}
    E_{edge}^{comm}(t) = (2^{d^{off}_edge (t)/W*\tau} - 1)* N_0W/\zeta * \tau
\end{equation}

As the PoIs are predefined and the I UAV s follow a
predetermined path, the energy consumed for the movement of I UAV s are not taken into consideration.
Given the energy of the system, our goal is to find
the optimal parameter values so as to minimize the expected cumulative energy across the time horizon. The
system policy in every time slot t can be given by X(t) =
Fedge(t), pi(t), Pedge(t), Sedge(t)}. Hence, the end-to-end
data offloading policy parameters X(t) aims at minimizinging total expected energy of the system. As the channel information for the data offloading is not deterministic and varies in the environment, the amount of bits
arrived at the Edge UAV depends upon the channel characteristics as well as the current position of the selected
I UAVi
. Such time-coupling of variables is responsible for
the stochastic nature of the system. The overall optimization model for the stable system performance is given as:
\section{Transmission energy optimization of Edge UAV}
The second sub-problem deals with the optimization of the
Edge UAV parameters for the amount of data offloaded
to the cloud. Further, here we can ignore the processor
frequency parameters and the associated constraints from
the optimization as they do not affect the energy optimization. The updated optimization model is given as The second sub-problem deals with the optimization of the
Edge UAV parameters for the amount of data offloaded
to the cloud. Further, here we can ignore the processor
frequency parameters and the associated constraints from
the optimization as they do not affect the energy optimization. The updated optimization model is given The overall solution approach of the proposed heterogeneous multi-UAV framework is given in Algorithm

\includegraphics[width=0.25\textwidth]{fig1.png}
\includegraphics[width=0.25\textwidth]{Fig2.png}

\section{Experimentation}
In this section, we present the simulation setup to validate the efficacy of our proposed Distance and Latency
Aware Trajectory Optimization with Lyapunov based system utility. The pre-computed trajectories of each of the
I UAVi are shared with the Edge UAV before the simulation starts. The simulation parameters are listed in
Table 1.
We have considered a 100m x 100m square region with
PoIs at 2m distance and at heights ranging from 70m
to 80m. There are total 2500 PoIs in the region. We
sample 500 PoIs uniformly at random. Simulation were
performed using three I UAVi and one Edge UAV . All
















\bibliography{references}
\bibliographystyle{abbrv}
\end{multicols}


\end{document}
